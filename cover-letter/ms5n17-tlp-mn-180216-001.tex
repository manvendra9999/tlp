\documentclass{mu-soton-letter}
\usepackage{graphicx,amssymb,amsmath,amstext,hyperref,graphics,float,eurosym,geometry,lineno}
%\linenumbers
\hypersetup{
    colorlinks=true,        % false: boxed links; true: colored links
    linkcolor=black,         % color of internal links (change box color with linkbordercolor)    https://www.sharelatex.com/learn/Hyperlinks
    citecolor=black,        % color of links to bibliography
    filecolor=black,      % color of file links
    urlcolor=black           % color of external links
}
%%%%%%%%%%%%%%%%%%%%%%%%%%%%%%%%%%%%%%%%%%%%%%%%%%
%%%%%%%%%%%%%%%%%%%%%%%%%%%%%%%%%%%%%%%%%%%%%%%%%%

\signature{Marcel}

\begin{document}
\begin{letter}{Prof. Lucio Frydman\\
  Editor,\\
  Journal of Magnetic Resonance\\[2cm]
  \textbf{Manuscript on a Modular Microfluidic NMR Probe System}}
\vfill
\opening{Dear Lucio,
}
Please find enclosed our manuscript titled
``Modular transmission line probes for microfluidic nuclear
magnetic resonance spectroscopy and imaging'', which we submit for publication
in the Journal of Magnetic Resonance.

 We report a novel modular dual channel
NMR probe assembly which enables a wide range microfluidic NMR experiments,
including proton-detected double resonance methods.
We show that its performance in terms of sensitivity, RF homogeneity, and
resolution compares favourably with the best microfluidic NMR systems that
have been reported to date. At the same time, the
probe assembly is designed to keep the manufacturing process as simple
as possible. The probe structure is built entirely from easily accessible
materials such as Aluminium profiles, and the detector is implemented
in standard printed circuit board technology.
We hope that this will enable other researchers within the magnetic resonance
community who are interested in developing applications
of microfluidic NMR spectroscopy to build their own systems.

\closing{With best wishes,}

\end{letter}

\end{document}
