\documentclass[12pt, letterpaper, twoside]{article}
\usepackage{graphicx,amssymb,amsmath,amstext,hyperref,graphics,float,eurosym,geometry,lineno}
%\linenumbers
\hypersetup{
    colorlinks=true,        % false: boxed links; true: colored links
    linkcolor=black,         % color of internal links (change box color with linkbordercolor)    https://www.sharelatex.com/learn/Hyperlinks
    citecolor=black,        % color of links to bibliography
    filecolor=black,      % color of file links
    urlcolor=black           % color of external links
}
%%%%%%%%%%%%%%%%%%%%%%%%%%%%%%%%%%%%%%%%%%%%%%%%%%
%%%%%%%%%%%%%%%%%%%%%%%%%%%%%%%%%%%%%%%%%%%%%%%%%%
\setlength{\parskip}{\baselineskip}
\title{Cover letter Modular transmission line probes for microfluidic nuclear magnetic resonance spectroscopy and imaging}
\begin{document}

Dear Editor,

We would like to submit a manuscript by M. Sharma and M. Utz entitled ``Modular transmission line probes for microfluidic nuclear magnetic resonance spectroscopy and imaging" for possible publication in Journal of Magnetic Resonance.

In this manuscript, we report a novel modular dual channel NMR probe assembly for generic high sensitivity microfluidic NMR experiments. The probe assembly is designed to keep the manufacturing process as simple as possible and retaining generic NMR and microfluidic functionality. The sample is contained in a home made exchangeable microfluidic device. Microfluidic devices are made from cheap plastic materials with rapid prototyping techniques. The probe body is built from the readily accessible materials and the detector is made from PCB materials. Modularity of the detector enables the use of different detectors on the same probe base while the use of PCB materials ensures quick and easy design and manufacturing. Spectral resolution around 0.005 ppm at 600 MHz has been achieved with nLOD of 1.4 nmol sec$^{-1/2}$. Results of probe performance and applications are presented to demonstrate the potential of the current design.

The developed probe design could be useful for various microfluidic-nmr experiments like reaction monitoring, micro-imaging and metabolic studies of biological samples, protein structure determination and hyperpolarisation.

For this reason, we believe this paper fits well to the scope of your journal.

Thank you for your consideration!

Sincerely,
\end{document}
